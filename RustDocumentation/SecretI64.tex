\subsection{SecretI64}
This is the same as the $\sregint$ in the underlying SCALE virtual machine.

\subsubsection{Conversion of Data}
\func{SecretI64::from(a)}
This takes an integer value $a$ and loads it into a value
of type $\SecretI$.
You can also pass in an argument of a $\SecretModp$ or a $\SecretBit$.
\begin{lstlisting}
    let sa = SecretI64::from(ConstI32::<8>);

    let sb = SecretBit::from(false);
    let ssb = SecretI64::from(sb);

    // See the main SCALE manual for the semantics of this conversion
    let si = SecretModp::from(ConstI32::<10>);
    let ssi = SecretI64::from(si);
\end{lstlisting}


\subsubsection{Arithmetic}
\func{SecretI64::mult2(a,b)}
One can access a multiplication operation which produces a double
word output as follows:
\begin{lstlisting}
    let s64a = SecretI64::from(0x4000000000000000_i64+28731371);
    let s64b = SecretI64::from(0x4000000000000000_i64+985724131);
    let (high,low)=SecretI64::mult2(s64a, s64b);
\end{lstlisting}


\subsubsection{Bit Twiddling}

\func{a.set_bit(b,n)}
This sets the $n$-th bit of $a$ equal to $b$.
\begin{lstlisting}
    let bit = SecretBit::from(true);
    let sa = SecretI64::from(ConstI32::<0>);
    let sb=sa.set_bit(bit, ConstU32::<10>);
\end{lstlisting}

\func{a.get_bit(n)}
This returns the $n$-th bit of $a$ as a $\SecretBit$.
\begin{lstlisting}
    let sa = SecretI64::from(123121);
    let sb=sb.get_bit(ConstU32::<10>);
\end{lstlisting}

\subsubsection{Comparisons}
The following `operators' can be applied between two $\SecretI$ values
or a $\SecretI$ value and a $\ints$ value. The output is a $\SecretBit$
value.
As the result of the operator cannot be used in a conditional branch,
we use the member function notation for such `operators', as opposed
to the operator notation. Thus syntactically the programmer is less
likely to make a mistake.
\func{a.gt(x), a.ge(x), a.lt(x), a.le(x), a.gt(x), a.eq(x), a.ne(x)}

\noindent
The following give variants which compare just to zero.
\func{a.gtz(), a.gez(), a.ltz(), a.lez(), a.gtz(), a.eqz(), a.nez(x)}

\subsubsection{Other Functions}

\func{SecretI64::randomize()}
Produces a (secret) random number in the range $[0,\ldots,2^{64}-1]$.
This random number is the `same' for all players.
\begin{lstlisting}
   let a = SecretI64::randomize();
\end{lstlisting}

\subsubsection{Memory Access}
\func{a.store_in_mem(x)}
To store data into memory location $50$ say you need to execute,
recalling that memory is accessed by $32$ bit values.
\begin{lstlisting}
    unsafe{ sa.store_in_mem(ConstU32::<50>); }
\end{lstlisting}
To store to a variable location you use
\begin{lstlisting}
    let x: i64 = 50;
    unsafe{ sa.store_in_mem(x); }
\end{lstlisting}

\func{SecretI64::load_from_mem(x)}
To load data from memory you do
\begin{lstlisting}
    let sa=SecretI64::load_from_mem(ConstU32::<50>);
    let x: i64 = 50;
    let sb=SecretI64::load_from_mem(x);
\end{lstlisting}


\subsubsection{Testing Data}
\func{x.test()}
In the SCALE target this takes the $\SecretI$ value $x$,
applies a reveal operation to it, and then writes the
resulting $\regint$ value into memory.

In the Test target this takes a value stored in the $\regint$
memory saved on the last SCALE invocation, and compares it to
$x$. If the two values are the same it prints the value, and the
line number in the rust file where this was executed.
Otherwise it aborts.

\func{x.test_value(y)}
Test whether the value held in $x$ is the same as the value held in $y$.
This is the same as \verb|x.test()| except the value is compared to
$y$ and not the value emulated in the test environment.
