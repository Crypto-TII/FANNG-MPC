\subsection{i64}
The Rust type $\ints$ corresponds to the $\regint$ type in the underlying SCALE virtual machine.
You create a value of type $\ints$ (everything in this document also applies to \verb|u64|)
by appending \verb|_i64| to a number, like in \verb|42_i64|. Mostly you
do not need this postfix, but sometimes the compiler cannot figure out that you meant to use an
$\ints$ and may thus give you type mismatch errors because it used a fallback to \verb|i32|.

\subsubsection{Conversion of Data}
\func{i64::from(x)}
You can convert from a $\ClearModp$ value to an $\ints$.
\begin{lstlisting}
    let ca = ClearModp::from(10);
    let a=i64::from(ca);
\end{lstlisting}

\subsubsection{IO}
The IO class functions for $\ints$ datatypes can be access via the following commands:

\func{i64::input(x)}
Loads an $\ints$ from the channel $x$.
\begin{lstlisting}
    let a=i64::input(Channel::<10>);
\end{lstlisting}

\func{a.output(x)}
Writes an $\ints$ to channel $x$.
\begin{lstlisting}
   let v: i64 = 1;
   v.output(Channel::<10>);
\end{lstlisting}

\subsubsection{Comparisons}
The following `operators' can be applied between two $\ints$ values
or a $\ints$ value and a $\SecretI$ value. The output being an $\ints$
value if the two arguments are $\ints$, and a $\SecretBit$ value otherwise.
As the result of the operator (when a $\SecretBit$) cannot be used in a
conditional branch, we use the member function notation for such `operators', as opposed
to the operator notation. Thus syntactically the programmer is less
likely to make a mistake.
\func{a.gt(x), a.ge(x), a.lt(x), a.le(x), a.gt(x), a.eq(x), a.ne(x)}

\subsubsection{Other Functions}

\func{x.rand()}
Produces a pseudo-random number in the range $[0,\ldots,x-1]$.
Note, every player gets the same value; this is used for randomized
algorithms where the algorithm needs to make a random choice.
\begin{lstlisting}
   let r = x.rand();
\end{lstlisting}

\func{i64::randomize()}
Produces a random number in the range $[0,\ldots,2^{64}-1]$.
This random number is the `same' for all players.
\begin{lstlisting}
    let a = i64::randomize();
\end{lstlisting}

\subsubsection{Memory Access}
\func{a.store_in_mem(x)}
To store data into memory location $50$ say you need to execute,
recalling that memory is accessed by $32$ bit values.
\begin{lstlisting}
    unsafe{ a.store_in_mem(50_u32); }
\end{lstlisting}
To store to a variable location you need to ensure the $x$ really
is an $\ints$ by doing:
\begin{lstlisting}
    let x: i64 = 50;
    unsafe{ a.store_in_mem(x); }
\end{lstlisting}

\func{i64::load_from_mem(x)}
To load data from memory you do
\begin{lstlisting}
    let a=i64::load_from_mem(50_u32);
    let x: i64 = 50;
    let b=i64::load_from_mem(x);
\end{lstlisting}


\subsubsection{Testing Data}

\func{black_box(x)}
If you just use \verb|let a = 1| then the Rust compiler can optimize
the variable away. If you really want to treat $a$ as a \verb|regint|
value in the SCALE runtime then use
\begin{lstlisting}
  let a = black_box(1);
\end{lstlisting}


\func{x.test()}
In the SCALE target this writes the $\regint$ value $x$
into memory at the address equivalent to the line number in
which \verb|test| was invoked.

In the Test target this takes a value stored in the $\regint$
memory saved on the last SCALE invocation, and compares it to
$x$. If the two values are the same it prints the value, and the
line number in the rust file where this was executed.
Otherwise it aborts.

\func{x.test_mem(loc)}
This compares the value held in variable $x$ to the memory location
at position $loc$, a simple example being
\begin{lstlisting}
    let c: i64 = 8;
    unsafe{ c.store_in_mem(3_u32); }
    let d: i64 = 8;
    d.test_mem(3_u32);
\end{lstlisting}
On the SCALE target this is basically a no-op, on the Test target
it does this comparison.

\func{x.test_value(y)}
Test whether the value held in $x$ is the same as the value held in $y$.
This is the same as \verb|x.test()| except the value is compared to
$y$ and not the value emulated in the test environment.
