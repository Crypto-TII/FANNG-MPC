\subsection{ClearModp}
This is the same as the $\cint$ in the underlying SCALE virtual machine.

\subsubsection{Conversion of Data}
\func{ClearModp::from(a)}
This takes a constant $a$, a runtime integer $i64$ or a $\ints$
and loads it into a value of type $\ClearModp$.
\begin{lstlisting}
    let ca = CleadModp::from(ConstI32::<8>);
    // Works, but creates multiple assembly instructions.
    let cx = CleadModp::from(42);
\end{lstlisting}


\subsubsection{IO}
The IO class functions for $\ClearModp$ datatypes can be access via the following commands:

\func{ClearModp::input(x)}
Loads an $\ClearModp$ from the channel $x$.
\begin{lstlisting}
    let a=ClearModp::input(Channel::<10>);
\end{lstlisting}

\func{a.output(x)}
Writes an $\ClearModp$ to channel $x$.
\begin{lstlisting}
   let v = ClearModp::from(ConstI32::<1>);
   v.output(Channel::<10>);
\end{lstlisting}


\subsubsection{Other Operations}

\func{x.lengendre()}
Computes the legendre symbol of $x$ modulo the prime underlying the
$\ClearModp$ type
\begin{lstlisting}
   let l = x.legendre();
\end{lstlisting}


\func{x.digest()}
Computes a pseudo-random digest of the value $x$.
\begin{lstlisting}
   let d = x.digest();
\end{lstlisting}


\func{ClearModp::randomize()}
Produces a random number in the range $[0,\ldots,p-1]$,
where $p$ is the prime underlying the $\ClearModp$ type.
This random number is the `same' for all players.
\begin{lstlisting}
   let a = ClearModp::randomize();
\end{lstlisting}



\subsubsection{Memory Access}
\func{a.store_in_mem(x)}
To store data into memory location $50$ say you need to execute,
recalling that memory is accessed by $32$ bit values.
\begin{lstlisting}
    unsafe{ ca.store_in_mem(ConstU32::<50>); }
\end{lstlisting}
To store to a variable location you use
\begin{lstlisting}
    let x: i64 = 50;
    unsafe{ ca.store_in_mem(x); }
\end{lstlisting}

\func{ClearModp::load_from_mem(x)}
To load data from memory you do
\begin{lstlisting}
    let ca=ClearModp::load_from_mem(ConstU32::<50>);
    let x: i64 = 50;
    cb=ClearModp::load_from_mem(x);
\end{lstlisting}


\subsubsection{Testing Data}
\func{x.test()}
In the SCALE target this takes the $\ClearModp$ value $x$,
applies a reveal operation to it, and then writes the
resulting $\ClearModp$ value into memory.

In the Test target this takes a value stored in the $\ClearModp$
memory saved on the last SCALE invocation, and compares it to
$x$. If the two values are the same it prints the value, and the
line number in the rust file where this was executed.
Otherwise it aborts.

\func{x.test_mem(loc)}
This compares the value held in variable $x$ to the memory location
at position $loc$, a simple example being
\begin{lstlisting}
    let c = ClearModp::from(ConstI32::<8>);
    unsafe{ c.store_in_mem(ConstU32::<3>); }
    ClearModp::from(ConstI32::<8>).test_mem(ConstU32::<3>);
\end{lstlisting}
On the SCALE target this is basically a no-op, on the Test target
it does this comparison.

\func{x.test_value(y)}
Test whether the value held in $x$ is the same as the value held in $y$.
This is the same as \verb|x.test()| except the value is compared to
$y$ and not the value emulated in the test environment.
