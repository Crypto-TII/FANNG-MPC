\mainsection{Bit Oriented Protocols}

These operations all refer to bitwise operations for the mod p based datatype \emph{SecretModp}.

\subsection{binary operations}
These are the basic bitwise operations.

\func{or_op(a,b)}
To perform a bitwise OR-operation on two SecretModp values a and b, execute
\begin{lstlisting}
	let c: SecretModp = or_op(a,b);
\end{lstlisting}

\func{mul_op(a,b)}
To get the product of two SecretModp values a and b, execute
\begin{lstlisting}
        let c: SecretModp = mul_op(a,b);
\end{lstlisting}

\func{addition_op(a,b)}
To get the sum of two SecretModp values a and b, execute
\begin{lstlisting}
        let c: SecretModp = addition_op(a,b);
\end{lstlisting}

\func{xor_op(a,b)}
To get the bitwise XOR-operation of two SecretModp values a and b, execute
\begin{lstlisting}
        let c: SecretModp = xor_op(a,b);
\end{lstlisting}

\func{carry(a,b)}
To compute the carry propagation of two two-bit values [a_0, a_1] and [b_0, b_1], execute this function. The inputs should be provided as two size-two Arrays of SecretModp values. The output will have the same format.
\begin{lstlisting}
        let c: Array<SecretModp, 2> = carry(a,b);
\end{lstlisting}

\func{num_bits(x)}
This function returns the number of bits that is used in memory to store the variable x (of any type).
\begin{lstlisting}
	let l: usize = num_bits(x);
\end{lstlisting}

\func{ceil_log_2(x)}
This function returns the minimal amount of bits needed to represent the value of variable x (of type u32).
\begin{lstlisting}
	let a: u32 = ceil_log_2(x);
\end{lstlisting}

\func{two_power(n)}
This function can be used to compute the n'th power of two. Both in- and output are of type u64.
\begin{lstlisting}
	let pow: u64 = two_power(n);
\end{lstlisting}

\func{inv(x)}
To compute the modular inverse of a SecretModp variable x, do
\begin{lstlisting}
	let x_inv: SecretModp = Inv(x);
\end{lstlisting}

\func{bits(x, m)}
This returns the m (u64) MSBs of bit decomposition (Slice<ClearModp>, little endian) of ClearModp value x.
\begin{lstlisting}
	let xbits: Slice<ClearModp> = bits(x, m);
\end{lstlisting}

\func{get_primecompl(bitlen)}
This auxilliary function returns the two-s complement of the prime P, with respect to the bit size of P.
\begin{lstlisting}
	let p_compl: Slice<ClearModp> = get_primecompl(bitlen);
\end{lstlisting}

\subsection{Protocols}
This section contains protocols that often make use of Arrays or Slices of SecretModp variables. 
\func{kopl(op, s)}
This function recursively computes the op-operation on all elements of the SecretModp slice s. This op must be a binary operation, eg. or_op.
Argument s must be a slice. If you wish to enter an array, convert it by slicing it inline.
\begin{lstlisting}
	let op_s: SecretModp = kopl(xor_op, &s);
	let op_s: SecretModp = kopl(xor_op, &a.slice(..));
\end{lstlisting}

\func{kor(s)}
Because kopl(or_op, s) is very common, it also has its own shortcut.
\begin{lstlisting}
	let or_s: SecretModp = kor(&s);
\end{lstlisting}

\func{preopl(op, s)}
This function creates a slice containing all pre-ops of the input slice. The i'th element of the resulting slice will thus be equal to op(s_0, ... , s_i).
\begin{lstlisting}
	let prexor_s: Slice<SecretModp, K> = preopl(xor_op, &s);
\end{lstlisting}

\func{preoplb(op, s)}
There is a separate function preoplb that executes preopl for inputs of form Slice< Array<SecretModp, 2>, K>. The operation is expected to support inputs of this form. 
\begin{lstlisting}
	let pre_carry: Slice<Array<SecretModp, 2>,2> = preoplb(carry, &s);
\end{lstlisting}


\func{preor(s)}
Just like for kor, preopl(or_op, s) deserves a shortcut.
\begin{lstlisting}
	let preor_s: Slice<SecretModp, K> = preor(&s);
\end{lstlisting}

\func{sumbits(s)}
This function treats the binary input slice as if it were the bit-decomposition of a variable s, and returns this variable s. The bit at s_0 is assumed to be the least significant bit. The function also accepts non-binary slices.
\begin{lstlisting}
	let sumbits_s: SecretModp = sumbits(&s);
\end{lstlisting}

\func{prandint(k)}
Returns a random SecretModp integer in the interval [0, 2^k-1], where k is a u64 number. 
During testing, this random number will always be equal to 2^k-1, due to the deterministic nature of emulation.
\begin{lstlisting}
	let rand_x: SecretModp = prandint(k);
\end{lstlisting}

\func{prandm(k, m, kappa)}
Returns two random SecretModp integers: r and r'. It also returns the bitdecomposition of r' as an Slice<SecretModp>. The u64-type inputs k, m and kappa determine the interval in which r and r' will lie: r is in [0, 2^(k+kappa-m)-1], r' is in [0, 2^k-1]
During testing, these random numbers will always be equal to 2^k-1.
\begin{lstlisting}
	let p_rand_m: (SecretModp, SecretModp, Slice<SecretModp>) = prandm(k, m, kappa);
\end{lstlisting}

\func{carryoutaux(&s)}
This function is a subroutine of the carry propagation function CarryOut(). The input should be given as a Slice of 2-size SecretModp Arrays: Slice< Array<SecretModp, 2> >. The output will then be of type SecretModp. 
Carry propagation for two bit inputs is defined as
\begin{lstlisting}
	$\circ: {0,1}^2 \times {0,1}^2 \to {0,1}^2: 
	(b_0, b_1) \circ (a_0, a_1) \mapsto (a_0*b_0, b_1+a_1*b_0)$
\end{lstlisting}
An example of this function being called:
\begin{lstlisting}
	let carryoutaux_res: SecretModp = carryoutaux(&s);
\end{lstlisting}

\func{carryout(&sa, &sb, c)}
This function evaluates the carry propagation for the sum of bitdecomposed inputs, sa: &Slice<ClearModp> and sb: &Slice<SecretModp>, with initial carry: SecretModp c.
The restulting carry is represented by a SecretModp value. 
\begin{lstlisting}
	let carryout_res: SecretModp = carryout(&sa, &sb, c);
\end{lstlisting}

\func{bitadd(&abits, &bbits)}
This function adds two bit-decomposed values, being either SecretModp or ClearModp. Note that for the addition of two ClearModp values, more efficient algorithms exist.
The output will be a Slice that contains the Secretmodp bits of the result. The size of this slice is one more than the size of the input arrays.
This function uses the function PreOpLB.
\begin{lstlisting}
	let cbits: Slice<SecretModp> = BitAdd(&abits, &bbits);
\end{lstlisting}

\func{bitincrement(&abits)}
Bitincrement takes a SecretModp bitarray abits as only argument, and returns the SecretModp bitslice of (abits+1). The output is a slice, so that the size can be one greater than the size of the input. 
\begin{lstlisting}
	let a_plus_one: Slice<SecretModp> = bitincrement(&abits);
\end{lstlisting}

\func{BitLT(a, &bbits)}
This function evaluates the comparison a < bbits, where a is a ClearModp value, and bbits is the SecretModp bitslice of value b. The result is a SecretModp bit c, that indicates the veracity of the statement. 
\begin{lstlisting}
	let c: SecretModp = bitlt(a, &bbits);
\end{lstlisting}

\func{bitltfull(&abits, &bbits)}
This function evaluates the comparison abits < bbits, where both abits and bbits are Slices containing the bit decomposition of values a and b, as either SecretModp bits or ClearModp bits. 
Note that it should not be used for ClearModp-ClearModp comparison. Note also that for ClearModp-SecretModp comparison, the function bitlt() can be used.
The output is a SecretModp bit, indicating the result of the comparison.
\begin{lstlisting}
	let c: SecretModp = bitltfull(&abits, &bbits);
\end{lstlisting}

\func{bitdec(a, k, m, kappa)}
This function returns the m MSB of the bit-decomposition of SecretModp variable a, as a Slice<SecretModp>. 
This function requires a statisctical security gap kappa. k represents the bitsize of a.
\begin{lstlisting}
	let abits: Slice<SecretModp> = bitdec(a, k, m, kappa);
\end{lstlisting}

\func{bitdecfull(a)}
As an alternative to the function bitdec(), this version of bit decomposition does not require a statistical security gap.
When the prime consists of 64 bits or more, a similar protocol bitdecfullbig() is called. 
\begin{lstlisting}
	let abits: Slice<SecretModp> = bitdecfull(a);
\end{lstlisting}



